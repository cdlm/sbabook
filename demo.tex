\documentclass[openany]{sbabook}

\setmainlanguage{english}
\setotherlanguage{latin}

\usepackage{lipsum}

\usepackage{lua-visual-debug}

% this is just for drawing a green rule on every page and test the code
% it can be safely deleted
\usepackage[contents={},angle=0,scale=1]{background}
\usetikzlibrary{calc}
\backgroundsetup{contents={%
  \begin{tikzpicture}[overlay]
  \draw[green,thick] ( $ (current page.west) + (0,2.55cm) $ )-- ++(2\textwidth,0);
  \draw[orange,thick] ( $ (current page.west) + (0,4.65cm) $ )-- ++(2\textwidth,0);
\end{tikzpicture}}}
% end of the code drawing the line

\title{Titre du livre}
\author{Auteur
    \and Collaborateur
    \and Autre
    \andnext Secrétaire
\and Trésorier}

\begin{document}
\begin{titlingpage}
    \maketitle
\end{titlingpage}

\frontmatter

% \tableofcontents*
% \listoffigures
% \listoftables
% \lstlistoflistings

\mainmatter


\chapter{A small chapter in Latin}

\chapterprecis{Demonstrating simple paragraphs and section headings, as well as a rather long precis, so that it uses at least two lines, even in the table of contents.}

\sidepar{Testing the margin note layout, hopefully this much text is enough to see what it looks like.}%
\textlatin{\lipsum[1-2]}


\section{More placeholder text}

\textlatin{\lipsum[3]}


\subsection{And even more}

\textlatin{\lipsum[4]}


\subsubsection{And still more}

\sidepar{Testing the margin note layout, hopefully this much text is enough to see what it looks like.}%
\textlatin{\lipsum[5-6]}


\paragraph{And still more}

\textlatin{\lipsum[7-9]}


\section{More placeholder text}

\textlatin{\lipsum[3]}


\subsection{And even more}

\textlatin{\lipsum[4]}


\section{More placeholder text}

\textlatin{\lipsum[3]}

\begin{figure}
    A FRAMED\\
    FIGURE
    \caption{This is not a figure}
\end{figure}


\subsection{And even more}

\textlatin{\lipsum[4]}


%%% TODO fontes
% \book{Livre second}
% \part{Épisode premier}

\chapter{Another empty chapter, except for the figures, tables, and really long title}
\chapterprecis{Long titles happen, and the heading needs to support it, but hyphenation should be avoided and lines balanced.}

\begin{leftbar}
    \lipsum[2]
\end{leftbar}

\begin{script}[fortytwo]{A small method}
SomeClass>>method: arg with: arg
    ^ 42
\end{script}

Like any Smalltalk object, you can inspect or explore the ZnResponse object. You might be wondering how this response was actually transferred over the network. That is easy with Zn, as the key HTTP objects all implement \#writeOn: for this purpose.

\begin{script}{Making a connection}
| response |
response := (ZnClient new)
  url: 'http://zn.stfx.eu/zn/small.html';
  get;
  response.
response writeOn: Transcript.
Transcript flush.
\end{script}

If you have the Transcript open, you should see something like the following:

\begin{script}{The server's answer}
HTTP/1.1 200 OK
Date: Tue, 08 May 2012 19:00:25 GMT
Modification-Date: Thu, 10 Feb 2011 08:32:30 GMT
Content-Length: 113
Server: Zinc HTTP Components 1.0
Vary: Accept-Encoding
Content-Type: text/html;charset=utf-8

<html>
<head><title>Small</title></head>
<body><h1>Small</h1><p>This is a small HTML document</p></body>
</html>
\end{script}

\begin{table}[p]
    \begin{sidecaption}{A stupid table}
        \begin{tabular}{ccc}
            \toprule
            foo & bar & baz \\
            \midrule
            coffee & 42 & 1.25\,€ \\
            tea & 1 & 2,00\,€ \\
            \bottomrule
        \end{tabular}
    \end{sidecaption}
\end{table}

\begin{figure}[p]
    \begin{framed}
        A FRAMED\\
        FIGURE
    \end{framed}
    \caption{This is not a figure}
\end{figure}

\begin{figure}[p]
    \begin{sidecaption}{A little caption on the side}
        \begin{shaded}
            ANOTHER\\
            FRAMED\\
            FIGURE
        \end{shaded}
    \end{sidecaption}
\end{figure}


\chapter{When will the chapters stop?}
\chapterprecis{wheeennnnn?}

\begin{boxedverbatim}
Hello
  this is verbatim
with some indents using spaces
	and others using tabs
\end{boxedverbatim}

\chapter{Chapter without a precis}

But it does have some content, just to check the spacing.
But it does have some content, just to check the spacing.
But it does have some content, just to check the spacing.

\chapter{More chapters}
\chapterprecis{This is not boring at all.}

Some content, long enough to fill up a good part of the first line and hopefully the second one as well…

\chapter{More chapters}
\chapterprecis{This is not boring at all, testing layout variations depending on how many lines the content spans.}

Still some content, again long enough to fill up a good part of the first line and hopefully the second one as well…

\chapter{Chapter without a precis}

But it does have some content, just to check the spacing.
But it does have some content, just to check the spacing.
But it does have some content, just to check the spacing.

\chapter{More chapters}
\chapterprecis{This is not boring at all.}

Some content, long enough to fill up a good part of the first line and hopefully the second one as well…

\chapter{More chapters}
\chapterprecis{This is not boring at all, testing layout variations depending on how many lines the content spans. When the precis needs more than two lines, it starts pushing the text down.}

Still some content, again long enough to fill up a good part of the first line and hopefully the second one as well…

\chapter{More chapters}
\chapterprecis{This is not boring at all, testing layout variations depending on how many lines the content spans. When the precis needs more than two lines, it starts pushing the text down. And it it is really really long then it will push it even more.}

Still some content, again long enough to fill up a good part of the first line and hopefully the second one as well…

\chapter{More chapters}
\chapterprecis{This is not boring at all.}

\chapter{More chapters}
\chapterprecis{This is not boring at all.}

\chapter{More chapters}
\chapterprecis{This is not boring at all.}

\chapter{More chapters}
\chapterprecis{This is not boring at all.}

\backmatter

\end{document}
