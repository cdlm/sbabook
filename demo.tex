\documentclass{sbabook}

\setmainlanguage{english}
\setotherlanguage{latin}

\usepackage{lipsum}

\title{Titre du livre}
\author{Auteur
    \and Collaborateur
    \and Autre
    \andnext Secrétaire
\and Trésorier}

\begin{document}
\begin{titlingpage}
    \maketitle
\end{titlingpage}

\frontmatter

\tableofcontents
\listoffigures
\listoftables

\mainmatter


\chapter{A small chapter in Latin}

\chapterprecis{Demonstrating simple paragraphs and section headings, as well as a rather long precis.}

\sidepar{Testing the margin note layout, hopefully this much text is enough to see what it looks like.}%
\textlatin{\lipsum[1-2]}


\section{More placeholder text}

\textlatin{\lipsum[3]}


\subsection{And even more}

\textlatin{\lipsum[4]}


\subsubsection{And still more}

\sidepar{Testing the margin note layout, hopefully this much text is enough to see what it looks like.}%
\textlatin{\lipsum[5-6]}


\paragraph{And still more}

\textlatin{\lipsum[7-9]}


%%% TODO fontes
% \book{Livre second}
% \part{Épisode premier}

\chapter{Another empty chapter, except for the figures, tables, and really long title}
\chapterprecis{Long titles happen, and the heading needs to support it.}

\begin{leftbar}
    \lipsum[2]
\end{leftbar}

\begin{script}[fortytwo]{A small method}
SomeClass>>method: arg with: arg
    ^ 42
\end{script}

Like any Smalltalk object, you can inspect or explore the ZnResponse object. You might be wondering how this response was actually transferred over the network. That is easy with Zn, as the key HTTP objects all implement \#writeOn: for this purpose.

\begin{script}{Making a connection}
| response |
response := (ZnClient new)
  url: 'http://zn.stfx.eu/zn/small.html';
  get;
  response.
response writeOn: Transcript.
Transcript flush.
\end{script}

If you have the Transcript open, you should see something like the following:

\begin{script}{The server's answer}
HTTP/1.1 200 OK
Date: Tue, 08 May 2012 19:00:25 GMT
Modification-Date: Thu, 10 Feb 2011 08:32:30 GMT
Content-Length: 113
Server: Zinc HTTP Components 1.0
Vary: Accept-Encoding
Content-Type: text/html;charset=utf-8

<html>
<head><title>Small</title></head>
<body><h1>Small</h1><p>This is a small HTML document</p></body>
</html>
\end{script}

\begin{table}[p]
    \begin{sidecaption}{A stupid table}
        \begin{tabular}{ccc}
            \toprule
            foo & bar & baz \\
            \midrule
            coffee & 42 & 1.25\,€ \\
            tea & 1 & 2,00\,€ \\
            \bottomrule
        \end{tabular}
    \end{sidecaption}
\end{table}

\begin{figure}[p]
    \begin{framed}
        A FRAMED\\
        FIGURE
    \end{framed}
    \caption{This is not a figure}
\end{figure}

\begin{figure}[p]
    \begin{sidecaption}{A little caption on the side}
        \begin{shaded}
            ANOTHER\\
            FRAMED\\
            FIGURE
        \end{shaded}
    \end{sidecaption}
\end{figure}


\backmatter

\end{document}
