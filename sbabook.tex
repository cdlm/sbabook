\documentclass[english,twoside,openany,showtrims]{sbabook}

\stockustrade
\setbleed{.125in}
\setlrmarginsandblock{.75in}{*}{*}
\setulmarginsandblock{.75in}{.75in}{*}
\checkandfixthelayout
\typeoutlayout


\usepackage[useregional]{datetime2}

% extract info from git commit
% requires gitinfo2 2.0.7, and below fix needed until 2.0.8 released
\makeatletter
\let\THEDAY\@dtm@currentday
\let\THEMONTH\@dtm@currentmonth
\let\THEYEAR\@dtm@currentyear
\makeatother
\usepackage[local,dirty=*]{gitinfo2}
\usepackage{xstring}% \IfEq and \StrCut
% a couple extensions...
\makeatletter
\newcommand\gitCommitInfo{%
  \IfEq{\gitRel}{}
  {commit \texttt{\gitAbbrevHash\gitDirty}}
  {\IfEq{\gitRoff}{0}
    {release \gitRel}
    {modified since release \gitRel{} --- commit \texttt{\gitAbbrevHash\gitDirty}}}}
\newcommand\gitdate{\DTMusedate{gitdate}}
\makeatother

\setotherlanguage{latin}
\usepackage{lipsum}

\usepackage{multirow}
\usepackage{graphicx}

\usepackage{lstsmalltalk}

\usepackage{url} % define and apply style to URLs
\def\url@sfstyle{\def\UrlFont{\sf}}
\urlstyle{sf}
\usepackage[unicode,breaklinks,hidelinks]{hyperref} % undecorated hyperlinks

\title{the Square Bracket Associates\titlebreak{}
  \texorpdfstring{\protect\LaTeX}{LaTeX} book class}
\author{Damien Pollet}
\date{\gitdate\titlebreak[\smallskip]{ -- }\protect\gitCommitInfo}

\hypersetup{pdfinfo = {
    Title = {\thetitle},
    Author = {\theauthor},
    Keywords = {LaTeX, document class, book, typography}}}

\newcommand\hrefnote[2]{% inline linked text, plus URL in footnote
  % #1: URL, #2: link text
  \href{#1}{#2}\footnote{\url{#1}}}

\begin{document}

%%%%%%%%%%%%%%%%%%%%%%%%%%%%%%%%%%%%%%%%%%%%%%%%%%%%%%%%%%%%%%%%%%%%%% 
% Title page and colophon on verso
\maketitle
\pagestyle{titlingpage}
\thispagestyle{titlingpage} % \pagestyle does not work on the first one…

\cleartoverso
{\small

  Copyright 2015 by Damien Pollet.

  The contents of this book are protected under the Creative Commons
  Attribution-ShareAlike 3.0 Unported license.

  You are \textbf{free}:
  \begin{itemize}
  \item to \textbf{Share}: to copy, distribute and transmit the work,
  \item to \textbf{Remix}: to adapt the work,
  \end{itemize}

  Under the following conditions:
  \begin{description}
  \item[Attribution.] You must attribute the work in the manner specified by the
    author or licensor (but not in any way that suggests that they endorse you
    or your use of the work).
  \item[Share Alike.] If you alter, transform, or build upon this work, you may
    distribute the resulting work only under the same, similar or a compatible
    license.
  \end{description}

  For any reuse or distribution, you must make clear to others the
  license terms of this work. The best way to do this is with a link to
  this web page: \\
  \url{http://creativecommons.org/licenses/by-sa/3.0/}

  Any of the above conditions can be waived if you get permission from
  the copyright holder. Nothing in this license impairs or restricts the
  author's moral rights.

  \begin{center}
    \includegraphics[width=0.2\textwidth]{CreativeCommons-BY-SA.pdf}
  \end{center}

  Your fair dealing and other rights are in no way affected by the
  above. This is a human-readable summary of the Legal Code (the full
  license): \\
  \url{http://creativecommons.org/licenses/by-sa/3.0/legalcode}

  \vfill

  \begin{tabular}{@{}c@{\quad}l}
    \multirow{2}{*}{\includegraphics[width=2em]{sba-logo.pdf}}
    & Developed for the books published by Square Bracket Associates. \\
    & \url{http://squarebracketassociates.org} \\[\smallskipamount]
  \end{tabular}
}


\frontmatter
\pagestyle{plain}

%%%%%%%%%%%%%%%%%%%%%%%%%%%%%%%%%%%%%%%%%%%%%%%%%%%%%%%%%%%%%%%%%%%%%% 
\chapter*{About this document}

This document is both the showcase and documentation of the SBAbook document
class.
You are of course encouraged to check its code, as a source of concrete examples.

The primary goals of the SBAbook class are:
\begin{itemize}
\item to provide a nice design for the Square Bracket Associates books,
\item to work as an export target for
  \hrefnote{https://github.com/pillar-markup}{Pillar}, the documentation markup
  used in the \hrefnote{http://pharo.org}{Pharo project}.
\end{itemize}
It is therefore meant more as a convenience class than a generic book class.
In fact, SBAbook itself is pretty small; it relies heavily on the
\hrefnote{https://www.ctan.org/pkg/memoir}{Memoir class} and on a selection of
additional packages, which it parameterizes to implement a specific design, from
page layout and typography to semantic markup.

The result is a class that should work well for technical books and should still
be pretty flexible, but which also has a few requirements and makes technical
choices on its own.
This document will thus only describe the specifics of SBAbook and its requirements.


\tableofcontents*
\mainmatter


%%%%%%%%%%%%%%%%%%%%%%%%%%%%%%%%%%%%%%%%%%%%%%%%%%%%%%%%%%%%%%%%%%%%%% 
\chapter{Technical requirements}


\section{Motivations}

Reasons (fonts, packages)


\section{A pretty up-to-date TeXlive}

how to update
important packages


\section{Building with Lua\LaTeX}

how to run, latexmk


%%%%%%%%%%%%%%%%%%%%%%%%%%%%%%%%%%%%%%%%%%%%%%%%%%%%%%%%%%%%%%%%%%%%%% 
\chapter{Page layout and design}


\section{Font choices}

In technical writing, and especially using \LaTeX{}, it's tempting to try to use
fonts as a semantic markup of sorts; unfortunately, this only results in a
jumble of too many fonts and uselessly noisy paragraphs.
Granted, a technical book can never be as minimal as a novel typeset in a single
size, single weight roman font, but some restraint should help maintain a clear
visual hierarchy.
We've thus tried to keep to a minimal set of fonts with clearly defined roles.
Luckily, there has been several impressive
open-source fonts released in the last years, which made several combinations
possible; in the end, we settled on the three families shown in
Table~\ref{tab:fontRoles}.

The workhorse font is \hrefnote{http://software.sil.org/gentium/}{Gentium}; it's
compact, very legible, and not as cold as the fonts we usually see in \LaTeX{}
documents.
\hrefnote{https://www.google.com/fonts/specimen/Open+Sans}{Open Sans} is neutral
but friendly and legible in small sizes; its minimal look makes it distinctive
but not overpowering, making it perfect for captions, titles, and page
decorations.

\begin{table}[htb]
  \begin{tabular}{ll}
    \toprule
    \textnormal{Gentium -- \textit{Italic}} & Primary, paragraph text \\
    \textsf{Open Sans -- \textbf{Bold}}     & Structural and secondary text: titles, captions \\
    \texttt{Inconsolata}                    & Verbatim text and code \\
    \bottomrule
  \end{tabular}
  \caption{Fonts used in the document and their roles}% TODO align all captions the same
  \label{tab:fontRoles}
\end{table}


\section{Text layout}

ragged right
spaced paragraphs with no indent
hanging section numbers

\begin{figure}[tb]
  \includegraphics{CreativeCommons-BY-SA}
  \caption{A rather large representation of the icon for the Creative Commons
    license we at Square Bracket Associates use for our books.}
  \label{fig:cc-by-sa-icon}
\end{figure}

\section{Custom semantic markup}

chapterprecis
constraints on content


\section{Source code and listings}

Source code relies on the well-known \code{listings} package, and sets it up
both for mentions of source code elements inline in the text, and for code
listings at the level of paragraphs or as floats.


\paragraph{Inline source code}
For short mentions of source code in the middle of the text, use the
\code§\code|...|§ macro.
This macro is an alias to \code§\lstinline§, which works like verbatim: it uses
an arbitrary delimiter.
Using this macro with curly braces like \code§\code{this}§ is possible, but this
is an experimental feature of \code{listings}, and might break in some cases.


\paragraph{Displayed source code}
For multi-line excerpts of source code that should appear in the flow of the
text, use the \code{listing} environment.
This environment uses the \code{tcolorbox} package with \code{listings}; this
provides the bracket-shaped decoration in the margin.

\begin{listing}{[Visual]Basic}
10 PRINT "Hello SBAbook!"
20 PRINT "This is a paragraph-level listing."
30 GOTO 10
\end{listing}


\paragraph{Referenceable listings}
If you need to reference a source code listing, it needs to have label like the
listing~\ref{lst:fooNewWith} below.

\begin{listing}[%
  title={\textbf{Listing~\thetcbcounter:} A factory method in class \code|Foo|},
  label=lst:fooNewWith,
  coltitle=black,
  attach boxed title to top left={xshift=-2cm},% FIXME
  boxed title style={empty},
]{Smalltalk}
Foo class>>with: parameter
  ^ self new
    initializeWith: parameter
\end{listing}

\paragraph{Floating listings}



\section{Packages and conventions}



%%%%%%%%%%%%%%%%%%%%%%%%%%%%%%%%%%%%%%%%%%%%%%%%%%%%%%%%%%%%%%%%%%%%%% 
\clearpage
\textlatin{\lipsum[2-7]}

\end{document}
